% Options for packages loaded elsewhere
\PassOptionsToPackage{unicode}{hyperref}
\PassOptionsToPackage{hyphens}{url}
%
\documentclass[
]{article}
\usepackage{amsmath,amssymb}
\usepackage{lmodern}
\usepackage{iftex}
\ifPDFTeX
  \usepackage[T1]{fontenc}
  \usepackage[utf8]{inputenc}
  \usepackage{textcomp} % provide euro and other symbols
\else % if luatex or xetex
  \usepackage{unicode-math}
  \defaultfontfeatures{Scale=MatchLowercase}
  \defaultfontfeatures[\rmfamily]{Ligatures=TeX,Scale=1}
\fi
% Use upquote if available, for straight quotes in verbatim environments
\IfFileExists{upquote.sty}{\usepackage{upquote}}{}
\IfFileExists{microtype.sty}{% use microtype if available
  \usepackage[]{microtype}
  \UseMicrotypeSet[protrusion]{basicmath} % disable protrusion for tt fonts
}{}
\makeatletter
\@ifundefined{KOMAClassName}{% if non-KOMA class
  \IfFileExists{parskip.sty}{%
    \usepackage{parskip}
  }{% else
    \setlength{\parindent}{0pt}
    \setlength{\parskip}{6pt plus 2pt minus 1pt}}
}{% if KOMA class
  \KOMAoptions{parskip=half}}
\makeatother
\usepackage{xcolor}
\IfFileExists{xurl.sty}{\usepackage{xurl}}{} % add URL line breaks if available
\IfFileExists{bookmark.sty}{\usepackage{bookmark}}{\usepackage{hyperref}}
\hypersetup{
  pdftitle={HEART DISEASE ANALYSIS},
  pdfauthor={Aryan Khokhar},
  hidelinks,
  pdfcreator={LaTeX via pandoc}}
\urlstyle{same} % disable monospaced font for URLs
\usepackage[margin=1in]{geometry}
\usepackage{color}
\usepackage{fancyvrb}
\newcommand{\VerbBar}{|}
\newcommand{\VERB}{\Verb[commandchars=\\\{\}]}
\DefineVerbatimEnvironment{Highlighting}{Verbatim}{commandchars=\\\{\}}
% Add ',fontsize=\small' for more characters per line
\usepackage{framed}
\definecolor{shadecolor}{RGB}{248,248,248}
\newenvironment{Shaded}{\begin{snugshade}}{\end{snugshade}}
\newcommand{\AlertTok}[1]{\textcolor[rgb]{0.94,0.16,0.16}{#1}}
\newcommand{\AnnotationTok}[1]{\textcolor[rgb]{0.56,0.35,0.01}{\textbf{\textit{#1}}}}
\newcommand{\AttributeTok}[1]{\textcolor[rgb]{0.77,0.63,0.00}{#1}}
\newcommand{\BaseNTok}[1]{\textcolor[rgb]{0.00,0.00,0.81}{#1}}
\newcommand{\BuiltInTok}[1]{#1}
\newcommand{\CharTok}[1]{\textcolor[rgb]{0.31,0.60,0.02}{#1}}
\newcommand{\CommentTok}[1]{\textcolor[rgb]{0.56,0.35,0.01}{\textit{#1}}}
\newcommand{\CommentVarTok}[1]{\textcolor[rgb]{0.56,0.35,0.01}{\textbf{\textit{#1}}}}
\newcommand{\ConstantTok}[1]{\textcolor[rgb]{0.00,0.00,0.00}{#1}}
\newcommand{\ControlFlowTok}[1]{\textcolor[rgb]{0.13,0.29,0.53}{\textbf{#1}}}
\newcommand{\DataTypeTok}[1]{\textcolor[rgb]{0.13,0.29,0.53}{#1}}
\newcommand{\DecValTok}[1]{\textcolor[rgb]{0.00,0.00,0.81}{#1}}
\newcommand{\DocumentationTok}[1]{\textcolor[rgb]{0.56,0.35,0.01}{\textbf{\textit{#1}}}}
\newcommand{\ErrorTok}[1]{\textcolor[rgb]{0.64,0.00,0.00}{\textbf{#1}}}
\newcommand{\ExtensionTok}[1]{#1}
\newcommand{\FloatTok}[1]{\textcolor[rgb]{0.00,0.00,0.81}{#1}}
\newcommand{\FunctionTok}[1]{\textcolor[rgb]{0.00,0.00,0.00}{#1}}
\newcommand{\ImportTok}[1]{#1}
\newcommand{\InformationTok}[1]{\textcolor[rgb]{0.56,0.35,0.01}{\textbf{\textit{#1}}}}
\newcommand{\KeywordTok}[1]{\textcolor[rgb]{0.13,0.29,0.53}{\textbf{#1}}}
\newcommand{\NormalTok}[1]{#1}
\newcommand{\OperatorTok}[1]{\textcolor[rgb]{0.81,0.36,0.00}{\textbf{#1}}}
\newcommand{\OtherTok}[1]{\textcolor[rgb]{0.56,0.35,0.01}{#1}}
\newcommand{\PreprocessorTok}[1]{\textcolor[rgb]{0.56,0.35,0.01}{\textit{#1}}}
\newcommand{\RegionMarkerTok}[1]{#1}
\newcommand{\SpecialCharTok}[1]{\textcolor[rgb]{0.00,0.00,0.00}{#1}}
\newcommand{\SpecialStringTok}[1]{\textcolor[rgb]{0.31,0.60,0.02}{#1}}
\newcommand{\StringTok}[1]{\textcolor[rgb]{0.31,0.60,0.02}{#1}}
\newcommand{\VariableTok}[1]{\textcolor[rgb]{0.00,0.00,0.00}{#1}}
\newcommand{\VerbatimStringTok}[1]{\textcolor[rgb]{0.31,0.60,0.02}{#1}}
\newcommand{\WarningTok}[1]{\textcolor[rgb]{0.56,0.35,0.01}{\textbf{\textit{#1}}}}
\usepackage{graphicx}
\makeatletter
\def\maxwidth{\ifdim\Gin@nat@width>\linewidth\linewidth\else\Gin@nat@width\fi}
\def\maxheight{\ifdim\Gin@nat@height>\textheight\textheight\else\Gin@nat@height\fi}
\makeatother
% Scale images if necessary, so that they will not overflow the page
% margins by default, and it is still possible to overwrite the defaults
% using explicit options in \includegraphics[width, height, ...]{}
\setkeys{Gin}{width=\maxwidth,height=\maxheight,keepaspectratio}
% Set default figure placement to htbp
\makeatletter
\def\fps@figure{htbp}
\makeatother
\setlength{\emergencystretch}{3em} % prevent overfull lines
\providecommand{\tightlist}{%
  \setlength{\itemsep}{0pt}\setlength{\parskip}{0pt}}
\setcounter{secnumdepth}{-\maxdimen} % remove section numbering
\ifLuaTeX
  \usepackage{selnolig}  % disable illegal ligatures
\fi

\title{HEART DISEASE ANALYSIS}
\author{Aryan Khokhar}
\date{2023-06-28}

\begin{document}
\maketitle

\hypertarget{introduction}{%
\subsection{INTRODUCTION}\label{introduction}}

This dataset is about heart disease. The dataset used for this analysis
can be downloaded from
\url{https://www.kaggle.com/datasets/johnsmith88/heart-disease-dataset}.

Before beginning, let me first lay the roadmap for this analysis. So,
I'll perform the following tasks on this dataset:\\
* Cleaning the data (if needed)\\
* Transforming the data OR Data Manipulation\\
* Analysis\\
* Conclusion

Starting with installing few packages and loading the dataset and also
removing all the previously stored variables.

\begin{Shaded}
\begin{Highlighting}[]
\FunctionTok{rm}\NormalTok{(}\AttributeTok{list=}\FunctionTok{ls}\NormalTok{()) }\CommentTok{\#Removes all previously stored variables}
\end{Highlighting}
\end{Shaded}

\begin{Shaded}
\begin{Highlighting}[]
\FunctionTok{library}\NormalTok{(tidyverse)}
\end{Highlighting}
\end{Shaded}

\begin{verbatim}
## -- Attaching core tidyverse packages ------------------------ tidyverse 2.0.0 --
## v dplyr     1.1.2     v readr     2.1.4
## v forcats   1.0.0     v stringr   1.5.0
## v ggplot2   3.4.2     v tibble    3.2.1
## v lubridate 1.9.2     v tidyr     1.3.0
## v purrr     1.0.1     
## -- Conflicts ------------------------------------------ tidyverse_conflicts() --
## x dplyr::filter() masks stats::filter()
## x dplyr::lag()    masks stats::lag()
## i Use the conflicted package (<http://conflicted.r-lib.org/>) to force all conflicts to become errors
\end{verbatim}

\begin{Shaded}
\begin{Highlighting}[]
\NormalTok{data }\OtherTok{\textless{}{-}} \FunctionTok{read.csv}\NormalTok{(}\StringTok{"heart.csv"}\NormalTok{)}
\FunctionTok{view}\NormalTok{(data)}
\end{Highlighting}
\end{Shaded}

The dataset has 14 variables(columns) and 1025 observations(rows). Here
we can use few other functions to get a brief summary about the data.

Here is the name of the columns and the explanation of each variable as
described in Kaggle.

\begin{enumerate}
\def\labelenumi{\arabic{enumi}.}
\tightlist
\item
  age: The age of a person\\
\item
  sex: The person's gender(1 = male, 0 = female)\\
\item
  cp: The types of chest pain experienced (Value 1: typical angina,
  Value 2: atypical angina, Value 3: non-anginal pain, Value 4:
  asymptomatic)\\
\item
  trestbps: Resting blood pressure (mm Hg on admission to the
  hospital)\\
\item
  chol: Cholesterol measurement in mg/dl\\
\item
  fbs: Fasting blood sugar (if \textgreater{} 120 mg/dl, 1 = true; 0 =
  false)\\
\item
  restecg: Resting electrocardiographic measurement (0 = normal, 1 =
  having ST-T wave abnormality, 2 = showing probable or definite left
  ventricular hypertrophy by Estes' criteria)\\
\item
  thalach: Maximum heart rate achieved\\
\item
  exang: Exercise induced angina (1 = yes; 0 = no)\\
\item
  oldpeak: ST depression induced by exercise relative to rest (`ST'
  relates to positions on the ECG plot)\\
\item
  slope: the slope of the peak exercise ST segment (Value 1: upsloping,
  Value 2: flat, Value 3: downsloping)\\
\item
  ca: The number of major vessels (0--3)\\
\item
  thal: A blood disorder called thalassemia (1 = normal; 2 = fixed
  defect; 3 = reversable defect)\\
\item
  target: Heart disease (0 = no, 1 = yes)
\end{enumerate}

\begin{Shaded}
\begin{Highlighting}[]
\FunctionTok{head}\NormalTok{(data) }\CommentTok{\#Shows first 6 rows of the data}
\end{Highlighting}
\end{Shaded}

\begin{verbatim}
##   age sex cp trestbps chol fbs restecg thalach exang oldpeak slope ca thal
## 1  52   1  0      125  212   0       1     168     0     1.0     2  2    3
## 2  53   1  0      140  203   1       0     155     1     3.1     0  0    3
## 3  70   1  0      145  174   0       1     125     1     2.6     0  0    3
## 4  61   1  0      148  203   0       1     161     0     0.0     2  1    3
## 5  62   0  0      138  294   1       1     106     0     1.9     1  3    2
## 6  58   0  0      100  248   0       0     122     0     1.0     1  0    2
##   target
## 1      0
## 2      0
## 3      0
## 4      0
## 5      0
## 6      1
\end{verbatim}

\begin{Shaded}
\begin{Highlighting}[]
\FunctionTok{tail}\NormalTok{(data) }\CommentTok{\#Shows last 6 rows of the data}
\end{Highlighting}
\end{Shaded}

\begin{verbatim}
##      age sex cp trestbps chol fbs restecg thalach exang oldpeak slope ca thal
## 1020  47   1  0      112  204   0       1     143     0     0.1     2  0    2
## 1021  59   1  1      140  221   0       1     164     1     0.0     2  0    2
## 1022  60   1  0      125  258   0       0     141     1     2.8     1  1    3
## 1023  47   1  0      110  275   0       0     118     1     1.0     1  1    2
## 1024  50   0  0      110  254   0       0     159     0     0.0     2  0    2
## 1025  54   1  0      120  188   0       1     113     0     1.4     1  1    3
##      target
## 1020      1
## 1021      1
## 1022      0
## 1023      0
## 1024      1
## 1025      0
\end{verbatim}

\begin{Shaded}
\begin{Highlighting}[]
\FunctionTok{glimpse}\NormalTok{(data)}
\end{Highlighting}
\end{Shaded}

\begin{verbatim}
## Rows: 1,025
## Columns: 14
## $ age      <int> 52, 53, 70, 61, 62, 58, 58, 55, 46, 54, 71, 43, 34, 51, 52, 3~
## $ sex      <int> 1, 1, 1, 1, 0, 0, 1, 1, 1, 1, 0, 0, 0, 1, 1, 0, 0, 1, 0, 1, 1~
## $ cp       <int> 0, 0, 0, 0, 0, 0, 0, 0, 0, 0, 0, 0, 1, 0, 0, 1, 2, 0, 1, 2, 2~
## $ trestbps <int> 125, 140, 145, 148, 138, 100, 114, 160, 120, 122, 112, 132, 1~
## $ chol     <int> 212, 203, 174, 203, 294, 248, 318, 289, 249, 286, 149, 341, 2~
## $ fbs      <int> 0, 1, 0, 0, 1, 0, 0, 0, 0, 0, 0, 1, 0, 0, 1, 0, 0, 0, 0, 1, 0~
## $ restecg  <int> 1, 0, 1, 1, 1, 0, 2, 0, 0, 0, 1, 0, 1, 1, 1, 1, 0, 0, 1, 0, 0~
## $ thalach  <int> 168, 155, 125, 161, 106, 122, 140, 145, 144, 116, 125, 136, 1~
## $ exang    <int> 0, 1, 1, 0, 0, 0, 0, 1, 0, 1, 0, 1, 0, 1, 1, 0, 0, 1, 0, 0, 0~
## $ oldpeak  <dbl> 1.0, 3.1, 2.6, 0.0, 1.9, 1.0, 4.4, 0.8, 0.8, 3.2, 1.6, 3.0, 0~
## $ slope    <int> 2, 0, 0, 2, 1, 1, 0, 1, 2, 1, 1, 1, 2, 1, 1, 2, 2, 1, 2, 2, 1~
## $ ca       <int> 2, 0, 0, 1, 3, 0, 3, 1, 0, 2, 0, 0, 0, 3, 0, 0, 1, 1, 0, 0, 0~
## $ thal     <int> 3, 3, 3, 3, 2, 2, 1, 3, 3, 2, 2, 3, 2, 3, 0, 2, 2, 3, 2, 2, 2~
## $ target   <int> 0, 0, 0, 0, 0, 1, 0, 0, 0, 0, 1, 0, 1, 0, 0, 1, 1, 0, 1, 1, 0~
\end{verbatim}

\begin{Shaded}
\begin{Highlighting}[]
\FunctionTok{summary}\NormalTok{(data) }\CommentTok{\#Statistical overview of the data}
\end{Highlighting}
\end{Shaded}

\begin{verbatim}
##       age             sex               cp            trestbps    
##  Min.   :29.00   Min.   :0.0000   Min.   :0.0000   Min.   : 94.0  
##  1st Qu.:48.00   1st Qu.:0.0000   1st Qu.:0.0000   1st Qu.:120.0  
##  Median :56.00   Median :1.0000   Median :1.0000   Median :130.0  
##  Mean   :54.43   Mean   :0.6956   Mean   :0.9424   Mean   :131.6  
##  3rd Qu.:61.00   3rd Qu.:1.0000   3rd Qu.:2.0000   3rd Qu.:140.0  
##  Max.   :77.00   Max.   :1.0000   Max.   :3.0000   Max.   :200.0  
##       chol          fbs            restecg          thalach     
##  Min.   :126   Min.   :0.0000   Min.   :0.0000   Min.   : 71.0  
##  1st Qu.:211   1st Qu.:0.0000   1st Qu.:0.0000   1st Qu.:132.0  
##  Median :240   Median :0.0000   Median :1.0000   Median :152.0  
##  Mean   :246   Mean   :0.1493   Mean   :0.5298   Mean   :149.1  
##  3rd Qu.:275   3rd Qu.:0.0000   3rd Qu.:1.0000   3rd Qu.:166.0  
##  Max.   :564   Max.   :1.0000   Max.   :2.0000   Max.   :202.0  
##      exang           oldpeak          slope             ca        
##  Min.   :0.0000   Min.   :0.000   Min.   :0.000   Min.   :0.0000  
##  1st Qu.:0.0000   1st Qu.:0.000   1st Qu.:1.000   1st Qu.:0.0000  
##  Median :0.0000   Median :0.800   Median :1.000   Median :0.0000  
##  Mean   :0.3366   Mean   :1.072   Mean   :1.385   Mean   :0.7541  
##  3rd Qu.:1.0000   3rd Qu.:1.800   3rd Qu.:2.000   3rd Qu.:1.0000  
##  Max.   :1.0000   Max.   :6.200   Max.   :2.000   Max.   :4.0000  
##       thal           target      
##  Min.   :0.000   Min.   :0.0000  
##  1st Qu.:2.000   1st Qu.:0.0000  
##  Median :2.000   Median :1.0000  
##  Mean   :2.324   Mean   :0.5132  
##  3rd Qu.:3.000   3rd Qu.:1.0000  
##  Max.   :3.000   Max.   :1.0000
\end{verbatim}

Since, our data is cleaned, so I'll directly move onto the next step
,i.e, Data Transformation.

\hypertarget{data-transformation}{%
\subsection{DATA TRANSFORMATION}\label{data-transformation}}

There are a lot of things that can be done with a dataset like this.
Also, it can be analyzed in so many different ways. So many different
plots and tables can be generated to explain in different ways.

To begin with, it will be helpful if we transform our data in another
table and extract and focus on those variables only which are more
important.

So, we'll make changes in
``target,sex,fbs,exang,cp,restecg,slope,ca,thal'' variables and leave
the rest as it is.

\begin{Shaded}
\begin{Highlighting}[]
\NormalTok{data2 }\OtherTok{\textless{}{-}}\NormalTok{ data }\SpecialCharTok{\%\textgreater{}\%} 
  \FunctionTok{mutate}\NormalTok{(}\AttributeTok{sex =} \FunctionTok{if\_else}\NormalTok{(sex }\SpecialCharTok{==} \DecValTok{1}\NormalTok{, }\StringTok{"MALE"}\NormalTok{, }\StringTok{"FEMALE"}\NormalTok{),}
         \AttributeTok{fbs =} \FunctionTok{if\_else}\NormalTok{(fbs }\SpecialCharTok{==} \DecValTok{1}\NormalTok{, }\StringTok{"\textgreater{}120"}\NormalTok{, }\StringTok{"\textless{}=120"}\NormalTok{),}
         \AttributeTok{exang =} \FunctionTok{if\_else}\NormalTok{(exang }\SpecialCharTok{==} \DecValTok{1}\NormalTok{, }\StringTok{"YES"}\NormalTok{, }\StringTok{"NO"}\NormalTok{),}
         \AttributeTok{cp =} \FunctionTok{if\_else}\NormalTok{(cp }\SpecialCharTok{==} \DecValTok{0}\NormalTok{, }\StringTok{"TYPICAL ANGINA"}\NormalTok{,}
                      \FunctionTok{if\_else}\NormalTok{(cp }\SpecialCharTok{==} \DecValTok{1}\NormalTok{,}\StringTok{"ATYPICAL ANGINA"}\NormalTok{,}
                       \FunctionTok{if\_else}\NormalTok{(cp }\SpecialCharTok{==} \DecValTok{2}\NormalTok{, }\StringTok{"NON{-}ANGINAL PAIN"}\NormalTok{, }\StringTok{"ASYMPTOMATIC"}\NormalTok{))),}
         \AttributeTok{restecg =} \FunctionTok{if\_else}\NormalTok{(restecg }\SpecialCharTok{==} \DecValTok{0}\NormalTok{, }\StringTok{"NORMAL"}\NormalTok{,}
                            \FunctionTok{if\_else}\NormalTok{(restecg }\SpecialCharTok{==} \DecValTok{1}\NormalTok{, }\StringTok{"ABNORMALITY"}\NormalTok{, }\StringTok{"PROBALBLE OR DEFINITE"}\NormalTok{)),}
         \AttributeTok{slope =} \FunctionTok{as.factor}\NormalTok{(slope),}
         \AttributeTok{ca =} \FunctionTok{as.factor}\NormalTok{(ca),}
         \AttributeTok{thal =} \FunctionTok{as.factor}\NormalTok{(thal),}
         \AttributeTok{target =} \FunctionTok{if\_else}\NormalTok{(target }\SpecialCharTok{==} \DecValTok{1}\NormalTok{, }\StringTok{"YES"}\NormalTok{, }\StringTok{"NO"}\NormalTok{)}
\NormalTok{         ) }\SpecialCharTok{\%\textgreater{}\%} 
  \FunctionTok{mutate\_if}\NormalTok{(is.character,as.factor) }\SpecialCharTok{\%\textgreater{}\%} 
\NormalTok{  dplyr}\SpecialCharTok{::}\FunctionTok{select}\NormalTok{(target,sex,fbs,exang,cp,restecg,slope,ca,thal,}\FunctionTok{everything}\NormalTok{())}

\FunctionTok{view}\NormalTok{(data2)}
\FunctionTok{glimpse}\NormalTok{(data2)}
\end{Highlighting}
\end{Shaded}

\begin{verbatim}
## Rows: 1,025
## Columns: 14
## $ target   <fct> NO, NO, NO, NO, NO, YES, NO, NO, NO, NO, YES, NO, YES, NO, NO~
## $ sex      <fct> MALE, MALE, MALE, MALE, FEMALE, FEMALE, MALE, MALE, MALE, MAL~
## $ fbs      <fct> <=120, >120, <=120, <=120, >120, <=120, <=120, <=120, <=120, ~
## $ exang    <fct> NO, YES, YES, NO, NO, NO, NO, YES, NO, YES, NO, YES, NO, YES,~
## $ cp       <fct> TYPICAL ANGINA, TYPICAL ANGINA, TYPICAL ANGINA, TYPICAL ANGIN~
## $ restecg  <fct> ABNORMALITY, NORMAL, ABNORMALITY, ABNORMALITY, ABNORMALITY, N~
## $ slope    <fct> 2, 0, 0, 2, 1, 1, 0, 1, 2, 1, 1, 1, 2, 1, 1, 2, 2, 1, 2, 2, 1~
## $ ca       <fct> 2, 0, 0, 1, 3, 0, 3, 1, 0, 2, 0, 0, 0, 3, 0, 0, 1, 1, 0, 0, 0~
## $ thal     <fct> 3, 3, 3, 3, 2, 2, 1, 3, 3, 2, 2, 3, 2, 3, 0, 2, 2, 3, 2, 2, 2~
## $ age      <int> 52, 53, 70, 61, 62, 58, 58, 55, 46, 54, 71, 43, 34, 51, 52, 3~
## $ trestbps <int> 125, 140, 145, 148, 138, 100, 114, 160, 120, 122, 112, 132, 1~
## $ chol     <int> 212, 203, 174, 203, 294, 248, 318, 289, 249, 286, 149, 341, 2~
## $ thalach  <int> 168, 155, 125, 161, 106, 122, 140, 145, 144, 116, 125, 136, 1~
## $ oldpeak  <dbl> 1.0, 3.1, 2.6, 0.0, 1.9, 1.0, 4.4, 0.8, 0.8, 3.2, 1.6, 3.0, 0~
\end{verbatim}

The transformed data is ready. Let's dive into data visualisation.

\hypertarget{analysis}{%
\subsection{ANALYSIS}\label{analysis}}

We will start with a simple bar graph depicting the presence and absence
of heart disease in individuals. We'll count how many individuals are
suffering from heart disease and how many are not.

\begin{Shaded}
\begin{Highlighting}[]
\FunctionTok{ggplot}\NormalTok{(data2, }\FunctionTok{aes}\NormalTok{(}\AttributeTok{x=}\NormalTok{data2}\SpecialCharTok{$}\NormalTok{target, }\AttributeTok{fill=}\NormalTok{data2}\SpecialCharTok{$}\NormalTok{target))}\SpecialCharTok{+}
  \FunctionTok{geom\_bar}\NormalTok{()}\SpecialCharTok{+}
  \FunctionTok{xlab}\NormalTok{(}\StringTok{"Heart Disease"}\NormalTok{)}\SpecialCharTok{+}
  \FunctionTok{ylab}\NormalTok{(}\StringTok{"Count"}\NormalTok{)}\SpecialCharTok{+}
  \FunctionTok{ggtitle}\NormalTok{(}\StringTok{"Presence \& Absence of Heart Disease"}\NormalTok{)}\SpecialCharTok{+}
  \FunctionTok{scale\_fill\_discrete}\NormalTok{(}\AttributeTok{name=}\StringTok{"Heart Disease"}\NormalTok{, }\AttributeTok{labels=}\FunctionTok{c}\NormalTok{(}\StringTok{"Absence"}\NormalTok{,}\StringTok{"Presence"}\NormalTok{))}
\end{Highlighting}
\end{Shaded}

\includegraphics{HeartDiseaseAnalysis_RMarkdown_files/figure-latex/Bar Plot for target (heart disease)-1.pdf}

We can also show this result in percentage of individuals having and not
having heart disease.

\begin{Shaded}
\begin{Highlighting}[]
\FunctionTok{round}\NormalTok{(}\FunctionTok{prop.table}\NormalTok{(}\FunctionTok{table}\NormalTok{(data2}\SpecialCharTok{$}\NormalTok{target)),}\DecValTok{2}\NormalTok{) }\CommentTok{\#To check the \%age }
\end{Highlighting}
\end{Shaded}

\begin{verbatim}
## 
##   NO  YES 
## 0.49 0.51
\end{verbatim}

Now moving forward we want to see how the number of patients are
distributed in different age groups and to check that first we'll form
the `age\_grp' column using our original data. This is because here we
want the sum of `target' column and since, in transformed data `target'
column is factor variable that is why we are using the original data in
which `target' is an integer variable.

\begin{Shaded}
\begin{Highlighting}[]
\NormalTok{data}\SpecialCharTok{$}\NormalTok{age\_grp }\OtherTok{\textless{}{-}} \FunctionTok{cut}\NormalTok{(data}\SpecialCharTok{$}\NormalTok{age, }\AttributeTok{breaks=}\FunctionTok{seq}\NormalTok{(}\DecValTok{25}\NormalTok{, }\DecValTok{77}\NormalTok{, }\DecValTok{4}\NormalTok{))}

\NormalTok{target\_by\_age }\OtherTok{\textless{}{-}}\NormalTok{ data }\SpecialCharTok{\%\textgreater{}\%} 
                  \FunctionTok{group\_by}\NormalTok{(age\_grp) }\SpecialCharTok{\%\textgreater{}\%} 
                  \FunctionTok{summarise}\NormalTok{(}\AttributeTok{heart\_patients =} \FunctionTok{sum}\NormalTok{(target))}
\NormalTok{target\_by\_age}
\end{Highlighting}
\end{Shaded}

\begin{verbatim}
## # A tibble: 12 x 2
##    age_grp heart_patients
##    <fct>            <int>
##  1 (25,29]              4
##  2 (33,37]             20
##  3 (37,41]             50
##  4 (41,45]             82
##  5 (45,49]             43
##  6 (49,53]             87
##  7 (53,57]             80
##  8 (57,61]             52
##  9 (61,65]             53
## 10 (65,69]             35
## 11 (69,73]             14
## 12 (73,77]              6
\end{verbatim}

Now let's visualise our above findings so that it is easier to interpret
what we have obtained.

\begin{Shaded}
\begin{Highlighting}[]
\NormalTok{data }\SpecialCharTok{\%\textgreater{}\%} 
  \FunctionTok{group\_by}\NormalTok{(age\_grp) }\SpecialCharTok{\%\textgreater{}\%} 
  \FunctionTok{count}\NormalTok{() }\SpecialCharTok{\%\textgreater{}\%} 
  \FunctionTok{ggplot}\NormalTok{()}\SpecialCharTok{+}
  \FunctionTok{geom\_col}\NormalTok{(}\FunctionTok{aes}\NormalTok{(}\AttributeTok{x=}\NormalTok{age\_grp, }\AttributeTok{y=}\NormalTok{n), }\AttributeTok{fill=}\StringTok{\textquotesingle{}pink\textquotesingle{}}\NormalTok{)}\SpecialCharTok{+}
  \FunctionTok{ggtitle}\NormalTok{(}\StringTok{"Age Analysis"}\NormalTok{)}\SpecialCharTok{+}
  \FunctionTok{xlab}\NormalTok{(}\StringTok{"Age Group"}\NormalTok{)}\SpecialCharTok{+}
  \FunctionTok{ylab}\NormalTok{(}\StringTok{"Total Patients"}\NormalTok{)}
\end{Highlighting}
\end{Shaded}

\includegraphics{HeartDiseaseAnalysis_RMarkdown_files/figure-latex/Age Group Graph-1.pdf}

Further we can form a contingency table showing proportion of males and
females suffering from heart disease or not.

\begin{Shaded}
\begin{Highlighting}[]
\FunctionTok{round}\NormalTok{(}\FunctionTok{prop.table}\NormalTok{(}\FunctionTok{table}\NormalTok{(data2}\SpecialCharTok{$}\NormalTok{sex, data2}\SpecialCharTok{$}\NormalTok{target)), }\DecValTok{2}\NormalTok{)}
\end{Highlighting}
\end{Shaded}

\begin{verbatim}
##         
##            NO  YES
##   FEMALE 0.08 0.22
##   MALE   0.40 0.29
\end{verbatim}

We can also compare the blood pressure in male and female patients,
further comparing the results for different kinds of chest pain using a
boxplot.

\begin{Shaded}
\begin{Highlighting}[]
\NormalTok{data2 }\SpecialCharTok{\%\textgreater{}\%} 
  \FunctionTok{ggplot}\NormalTok{(}\FunctionTok{aes}\NormalTok{(}\AttributeTok{x=}\NormalTok{sex, }\AttributeTok{y=}\NormalTok{trestbps))}\SpecialCharTok{+}
  \FunctionTok{geom\_boxplot}\NormalTok{()}\SpecialCharTok{+}
  \FunctionTok{xlab}\NormalTok{(}\StringTok{"Sex"}\NormalTok{)}\SpecialCharTok{+}
  \FunctionTok{ylab}\NormalTok{(}\StringTok{"BP (mm Hg)"}\NormalTok{)}\SpecialCharTok{+}
  \FunctionTok{facet\_wrap}\NormalTok{(}\SpecialCharTok{\textasciitilde{}}\NormalTok{cp)}
\end{Highlighting}
\end{Shaded}

\includegraphics{HeartDiseaseAnalysis_RMarkdown_files/figure-latex/Compare blood pressure with chest pain-1.pdf}

Again let us make another boxplot and now compare the cholesterol level
on male and female and grouped by chest pain.

\begin{Shaded}
\begin{Highlighting}[]
\NormalTok{data2 }\SpecialCharTok{\%\textgreater{}\%} 
  \FunctionTok{ggplot}\NormalTok{(}\FunctionTok{aes}\NormalTok{(}\AttributeTok{x=}\NormalTok{sex, }\AttributeTok{y=}\NormalTok{chol))}\SpecialCharTok{+}
  \FunctionTok{geom\_boxplot}\NormalTok{()}\SpecialCharTok{+}
  \FunctionTok{xlab}\NormalTok{(}\StringTok{"Sex"}\NormalTok{)}\SpecialCharTok{+}
  \FunctionTok{ylab}\NormalTok{(}\StringTok{"Cholesterol (mg/dl)"}\NormalTok{)}\SpecialCharTok{+}
  \FunctionTok{facet\_wrap}\NormalTok{(}\SpecialCharTok{\textasciitilde{}}\NormalTok{cp)}
\end{Highlighting}
\end{Shaded}

\includegraphics{HeartDiseaseAnalysis_RMarkdown_files/figure-latex/Compare cholesterol with chest pain-1.pdf}

It will be great if we could see how the variables are correlated with
each other. For that we must install and load the ``corrplot'' package
as well as the ``ggplot2'' package.

\begin{Shaded}
\begin{Highlighting}[]
\CommentTok{\#install.packages("corrplot")}
\FunctionTok{library}\NormalTok{(corrplot)}
\end{Highlighting}
\end{Shaded}

\begin{verbatim}
## corrplot 0.92 loaded
\end{verbatim}

\begin{Shaded}
\begin{Highlighting}[]
\FunctionTok{library}\NormalTok{(ggplot2)}
\end{Highlighting}
\end{Shaded}

We'll start by finding out the correlation among last 5 columns of our
dataset ``data2'' then finally showcasing it using corrplot() function.

\begin{Shaded}
\begin{Highlighting}[]
\NormalTok{cor\_heart }\OtherTok{\textless{}{-}} \FunctionTok{round}\NormalTok{(}\FunctionTok{cor}\NormalTok{(data2[,}\DecValTok{10}\SpecialCharTok{:}\DecValTok{14}\NormalTok{]),}\DecValTok{2}\NormalTok{) }
\NormalTok{cor\_heart}
\end{Highlighting}
\end{Shaded}

\begin{verbatim}
##            age trestbps  chol thalach oldpeak
## age       1.00     0.27  0.22   -0.39    0.21
## trestbps  0.27     1.00  0.13   -0.04    0.19
## chol      0.22     0.13  1.00   -0.02    0.06
## thalach  -0.39    -0.04 -0.02    1.00   -0.35
## oldpeak   0.21     0.19  0.06   -0.35    1.00
\end{verbatim}

\begin{Shaded}
\begin{Highlighting}[]
\FunctionTok{corrplot}\NormalTok{(cor\_heart, }\AttributeTok{method =} \StringTok{"square"}\NormalTok{, }\AttributeTok{type =} \StringTok{"upper"}\NormalTok{)}
\end{Highlighting}
\end{Shaded}

\includegraphics{HeartDiseaseAnalysis_RMarkdown_files/figure-latex/Correlation-1.pdf}

\hypertarget{conclusion}{%
\subsection{CONCLUSION}\label{conclusion}}

\begin{itemize}
\tightlist
\item
  From our analysis, we can see that 51\% of the individuals are
  suffering from heart disease. But the difference is not that much
  large and therefore, we can conclude that the chances of getting
  infected by a heart disease is almost equal to that of not having any
  heart disease.\\
\item
  The age group 53-61 has maximum number of heart patients.\\
\item
  The contingency table shows that the chances of a female for having a
  heart disease is 22\% whereas for a male it is 29\%. That is males are
  less immune to heart related problems.
\item
  Males having `Asymptotic' chest pain(CP) have less bloop pressure(BP)
  than females and same can be concluded for `Atypical Angina' as well
  as `Typical Angina' CP. But in case of `Non-Anginal Pain' males tends
  to have slightly higher BP as compared to females.\\
\item
  However, males with `Atypical Angina', `Typical Angina' and
  `Non-Anginal' CP have higher cholesterol level but for `Asymptotic' CP
  both of them have same level of cholesterol level.\\
\item
  Lastly, from the correlation plot we can see that age of an individual
  and trestbps(resting blood pressure), chol(cholesterol level),
  oldpeak(ST depression induced by exercise relative to rest), are
  positively correlated, that is as the age of a person increases, these
  all values also increases.\\
  Whereas, the age and thalach(maximum heart rate achieved) are
  inversely correlated to each other.\\
  Also, thalach(maximum heart rate achieved) and oldpeak(ST depression
  induced by exercise relative to rest) are negatively correlated.
\end{itemize}

\end{document}
